\documentclass[12pt]{article}
\usepackage[top=1in,bottom=1in,left=0.75in,right=0.75in,centering]{geometry}
\usepackage{fancyhdr}
\usepackage{epsfig}
\usepackage[pdfborder={0 0 0}]{hyperref}
\usepackage{palatino}
\usepackage{wrapfig}
\usepackage{lastpage}
\usepackage{color}
\usepackage{ifthen}
\usepackage[table]{xcolor}
\usepackage{graphicx,type1cm,eso-pic,color}
\usepackage{hyperref}
\usepackage{amsmath}
\usepackage{wasysym}

\def\course{CS 2501: DSA2}
\def\homework{Divide and Conquer - Basic: Recurrence Relations}
\def\semester{Spring 2020}

\newboolean{solution}
\setboolean{solution}{false}

% add watermark if it's a solution exam
% see http://jeanmartina.blogspot.com/2008/07/latex-goodie-how-to-watermark-things-in.html
\makeatletter
\AddToShipoutPicture{%
\setlength{\@tempdimb}{.5\paperwidth}%
\setlength{\@tempdimc}{.5\paperheight}%
\setlength{\unitlength}{1pt}%
\put(\strip@pt\@tempdimb,\strip@pt\@tempdimc){%
\ifthenelse{\boolean{solution}}{
\makebox(0,0){\rotatebox{45}{\textcolor[gray]{0.95}%
{\fontsize{5cm}{3cm}\selectfont{\textsf{Solution}}}}}%
}{}
}}
\makeatother

\pagestyle{fancy}

\fancyhf{}
\lhead{\course}
\chead{Page \thepage\ of \pageref{LastPage}}
\rhead{\semester}
%\cfoot{\Large (the bubble footer is automatically inserted into this space)}

\setlength{\headheight}{14.5pt}

\newenvironment{itemlist}{
\begin{itemize}
\setlength{\itemsep}{0pt}
\setlength{\parskip}{0pt}}
{\end{itemize}}

\newenvironment{numlist}{
\begin{enumerate}
\setlength{\itemsep}{0pt}
\setlength{\parskip}{0pt}}
{\end{enumerate}}

\newcounter{pagenum}
\setcounter{pagenum}{1}
\newcommand{\pageheader}[1]{
\clearpage\vspace*{-0.4in}\noindent{\large\bf{Page \arabic{pagenum}: {#1}}}
\addtocounter{pagenum}{1}
\cfoot{}
}

\newcounter{quesnum}
\setcounter{quesnum}{1}
\newcommand{\question}[2][??]{
\begin{list}{\labelitemi}{\leftmargin=2em}
\item [\arabic{quesnum}.] {} {#2}
\end{list}
\addtocounter{quesnum}{1}
}


\definecolor{red}{rgb}{1.0,0.0,0.0}
\newcommand{\answer}[2][??]{
\ifthenelse{\boolean{solution}}{
\color{red} #2 \color{black}}
{\vspace*{#1}}
}

\definecolor{blue}{rgb}{0.0,0.0,1.0}

\begin{document}

\section*{\homework}

%----------------------------------------------------------------------
\noindent Directly solve, by unrolling the recurrence, the following relations to find their exact solutions.

\question[2]{
$T(n) = T(n-1) + n$
}

\answer[0 in]{
$\Theta(n^2)$
}

\question[2]{
$T(n) = T(\frac{n}{2})+1$
}

\answer[0 in]{
$\Theta(log(n))$
}

\vspace{12pt}

%----------------------------------------------------------------------

\noindent Use induction to show bounds on the following recurrence relations.

\question[2]{
Show that $T(n)=2T(\frac{n}{2})+n \in \Omega(nlog(n))$. \emph{Note: We are using big-omega here, so your inequality will use $T(n) \geq c*n*log(n)$}.
}

\answer[0 in]{
...
}

\question[2]{
Show that $T(n) = 2T(\frac{n}{2}+17)+n \in O(nlog(n))$.
}

\answer[0 in]{
...
}

\question[2]{
Show that $T(n)=2T(\sqrt{n})+log(n) \in O(log(n)*log(log(n)))$. \emph{Hint: Try creating a new variable m and substituting the equation for m to make it look like a common recurrence we've seen before. Then solve the easier recurrence and substitute n back in for m at the end.}
}

\answer[0 in]{
...
}

\question[2]{
Show that $T(n)=4T(\frac{n}{2})+n \in \Theta(n^2)$. You'll need to subtract off a lower-order term to make the induction work here.
}

\answer[0 in]{
...
}

\question[2]{
Show that $T(n)=4T(\frac{n}{3})+n \in \Theta(n^{log_3(4)})$. You'll need to subtract off a lower-order term to make the induction work here.
}

\answer[0 in]{
...
}

\vspace{12pt}

%----------------------------------------------------------------------

\noindent Use the master theorem (or main recurrence theorem if applicable) to solve the following recurrence relations. State which case of the theorem you are using and why.

\question[2]{
$T(n)=2T(\frac{n}{4})+1$
}

\answer[0 in]{
...
}

\question[2]{
$T(n)=2T(\frac{n}{4})+\sqrt{n}$
}

\answer[0 in]{
...
}

\question[2]{
$T(n)=2T(\frac{n}{4})+n$
}

\answer[0 in]{
...
}

\question[2]{
$T(n)=2T(\frac{n}{4})+n^2$
}

\answer[0 in]{
...
}
\end{document}
