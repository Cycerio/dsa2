\documentclass[10pt]{article}
\usepackage[top=1in,bottom=1in,left=0.75in,right=0.75in,centering]{geometry}
\usepackage{fancyhdr}
\usepackage{epsfig}
\usepackage[pdfborder={0 0 0}]{hyperref}
\usepackage{palatino}
\usepackage{wrapfig}
\usepackage{lastpage}
\usepackage[T1]{fontenc}
\usepackage{color}
\usepackage{ifthen}
\usepackage[table]{xcolor}
\usepackage{graphicx,type1cm,eso-pic,color}
\usepackage{amsmath}
\usepackage{wasysym}
\usepackage{algorithm}
\usepackage[noend]{algpseudocode}
\usepackage{array}
\usepackage[utf8]{inputenc}
\usepackage[english]{babel}
\usepackage{multicol}
\usepackage{enumitem}
\usepackage{listings}
\lstset{
	language=Java
}
\usepackage{subcaption}

\newcolumntype{M}[1]{>{\centering\arraybackslash}m{#1}}
\newcolumntype{N}{@{}m{0 pt}@{}}

\makeatletter
% Reinsert missing \algbackskip
\def\algbackskip{\hskip-\ALG@thistlm}
\makeatother

\def\course{CS 2501: DSA2}
\def\exam{Quiz Booklet}
\def\semester{Spring 2020}

\newboolean{solution}
\setboolean{solution}{false}

% add watermark if it's a solution exam
% see http://jeanmartina.blogspot.com/2008/07/latex-goodie-how-to-watermark-things-in.html
\makeatletter
\AddToShipoutPicture{%
\setlength{\@tempdimb}{.5\paperwidth}%
\setlength{\@tempdimc}{.5\paperheight}%
\setlength{\unitlength}{1pt}%
\put(\strip@pt\@tempdimb,\strip@pt\@tempdimc){%
\ifthenelse{\boolean{solution}}{
\makebox(0,0){\rotatebox{45}{\textcolor[gray]{0.95}%
{\fontsize{5cm}{3cm}\selectfont{\textsf{Solution}}}}}%
}{}
}}
\makeatother

\pagestyle{fancy}

\fancyhf{}
\lhead{\course\ \exam, \semester}
\chead{Page \thepage\ of \pageref{LastPage}}
\rhead{UVa userid: \hspace{0.75in}}
\cfoot{}

\setlength{\headheight}{14.5pt}

\newenvironment{itemlist}{
\begin{itemize}
\setlength{\itemsep}{0pt}
\setlength{\parskip}{0pt}}
{\end{itemize}}

\newenvironment{numlist}{
\begin{enumerate}
\setlength{\itemsep}{0pt}
\setlength{\parskip}{0pt}}
{\end{enumerate}}

\newcounter{pagenum}
\setcounter{pagenum}{2}
\newcommand{\pageheader}[1]{
\clearpage\vspace*{-0.4in}\noindent{\large\bf{{#1}}}
\addtocounter{pagenum}{1}
\cfoot{}
}

\newcounter{quesnum}
\setcounter{quesnum}{1}
\newcommand{\question}[2][??]{
\begin{list}{\labelitemi}{\leftmargin=2em}
\item [\arabic{quesnum}.] {[}{#1} points{]} {#2}
\end{list}
\addtocounter{quesnum}{1}
}

\definecolor{red}{rgb}{1.0,0.0,0.0}
\newcommand{\answer}[2][??]{
%\begin{tabular}{p{0.25in}p{6in}}

\ifthenelse{\boolean{solution}}{
\color{red} #2 \color{black}}
{\vspace*{#1} }
%\end{tabular}
}

\begin{document}

\section*{\course\ \exam}

\vspace{0.1in}

\vspace{0.35in}

\noindent This booklet contains question pools for the quizzes for this course. You should use this resource to study potential questions that will be asked on the quizzes.

\vspace{12pt}

\noindent DO NOT write on this booklet, as we may use it multiple times in future weeks.

\vspace{12pt}

\noindent There are \pageref{LastPage} pages to this quiz booklet.

\vspace{12pt}



\vspace{0.15in}

\begin{quotation}
\begin{centering}
%\noindent {\em In theory, there is no difference between theory and practice.\\But, in practice, there is.\\}
%\noindent {\em The Tao that is seen \\ Is not the true Tao, \\ until You bring fresh toner.\\}
\noindent {\em A crash reduces\\Your expensive computer\\To a simple stone.\\}
%\noindent {\em Three things are certain:\\Death, taxes, and lost data.\\Guess which has occurred.\\}
%\noindent {\em You step in the stream,\\But the water has moved on.\\This page is not here.\\}
%\noindent {\em Serious error.\\All shortcuts have disappeared.\\Screen. Mind. Both are blank.\\}
\end{centering}
\end{quotation}


%----------------------------------------------------------------------

\pageheader{Graphs - Basic}

\noindent \\
\textbf{Short Answer Questions}

\begin{enumerate}
	\setlength\itemsep{0.25em}
	\item Given an undirected graph with $V$ nodes (no loops allowed). What is the maximum number of edges the graph can have?
	\item Given a directed graph with $V$ nodes (no loops allowed). What is the maximum number of edges the graph can have?
	\item Name one advantage and one disadvantage of storing a graph as an adjacency matrix.
	\item Name on advantage and one disadvantage of storing a graph as an adjacency list.
	\item Describe one way we can handle storing costs of graph edges. Make sure to answer for an adjacency list AND an adjacency matrix.
	\item Does breadth-first search always find the shortest path between two nodes for an undirected, unweighted graph?
	\item Which has better time-complexity? BFS or DFS?
	\item Which has better space-complexity? BFS or DFS?
	\item What is the run-time of BFS? Explain your answer.
	\item What is the run-time of DFS? Explain your answer.
	\item Briefly describe how you might use DFS to count the number of disconnected components (disconnected sub-graphs) in a graph.
	\item What is the run-time of topological sort?
	\item What is the run-time of Dijkstra's Algorithm?
	\item Dijkstra's Algorithm may not work if given negative cost edges. Provide a counter-example to illustrate this.
\end{enumerate}

\vspace{0.5in}

\textbf{Coding Questions}
\begin{enumerate}
	\setlength\itemsep{0.25em}
	\item Psuedo-code a method that performs a breadth-first search on a graph. Assume each node stores a number num. Print out each num as you visit each node.
	\item Psuedo-code a method that performs a depth-first search on a graph. Assume each node stores a number num. Print out each num as you visit each node.
	\item Pseudo-code the topological sort algorithm.
	\item Psuedo-code Dijkstra's Algorithm.
\end{enumerate}


\newif\ifcomment
\commentfalse
\ifcomment

\fi

%----------------------------------------------------------------------

\pageheader{Graphs - Advanced}

\noindent \\
\textbf{Short Answer Questions}

\begin{enumerate}
	\setlength\itemsep{0.25em}
	\item Given a flow network and current flow values, produce the residual graph. A network to use will be drawn on the board.
	\item When discussing flow networks, what is backflow? Explain why it is necessary.
	\item Describe how the Ford-Fulkerson algorithm uses depth-first search. What purpose does it serve? Why not use breadth-first search?
	\item What is the run-time of the Ford-Fulkerson Algorithm? Explain your answer.
	\item Define the following terms regarding flow networks: Cut, capacity of a cut, and net-flow of a cut.
	\item What does the max-flow, min-cut theorem state? What does this say about algorithms for the two problems?
	\item What is a reduction? Why is it useful when comparing algorithms?
	\item Describe how you would solve for the maximum-flow of a network that has multiple sources and multiple sinks.
\end{enumerate}

\vspace{0.5in}

\textbf{Coding Questions}
\begin{enumerate}
	\setlength\itemsep{0.25em}
	\item Psuedo-code the Ford-Fulkerson Algorithm.
	\item Prove the flow-value lemma: The the net-flow across any cut is always equal to the flow $f$
	\item Psuedo-code an algorithm that solves the Bipartite matching problem.
\end{enumerate}


%----------------------------------------------------------------------

\pageheader{Find-Union: Prims and Kruskals}

\noindent \\
\textbf{Short Answer Questions}

\begin{enumerate}
	\setlength\itemsep{0.25em}
	\item Give an example of a graph (with edge costs) that has more than one minimum spanning tree.
	\item How many edges does a minimum spanning tree have as a function of the original connected graph $G=(V,E)$? Explain why.
	\item True or False: If the edge weights of a graph are all unique, then the minimum spanning tree is unique as well. Explain your answer.
	\item Given a graph (drawn on the board) and a starting node, step through Prim's algorithm. Draw the MST and list the order in which the nodes are added to the tree.
	\item What do we mean when we say that the Minimum Spanning Tree problem has optimal substructure?
	\item What is the runtime of Prim's algorithm?
	\item What is the runtime of Kruskal's algorithm?
	\item Describe how the findSet(int i) method works for a find-union data structure? What is the runtime?
	\item Describe how the union(int i, int j) method works for a find-union data structure. What is the runtime?
	\item When implementing a find-union structure, what is union by rank? Why is it useful?
	\item When implementing a find-union structure, what is path compression? Why is it useful?

\end{enumerate}

\vspace{0.5in}

\textbf{Coding Questions}
\begin{enumerate}
	\setlength\itemsep{0.25em}
	\item Write pseudo-code for Prim's algorithm
	\item Write pseudo-code for Kruskal's algorithm
	\item Pseudo-code the findSet(int i) method of a find-union structure.
	\item Pseudo-code the union(int i, int j) method of a find-union structure.

\end{enumerate}


%----------------------------------------------------------------------

\end{document}
